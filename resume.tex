%%%%%%%%%%%%%%%%%%%%%%%%%%%%%%%%%%%%%%%%%
% Freeman Curriculum Vitae
% XeLaTeX Template
% Version 3.0 (September 3, 2021)
%
% This template originates from:
% https://www.LaTeXTemplates.com
%
% Authors:
% Vel (vel@LaTeXTemplates.com)
% Alessandro Plasmati
%
% License:
% CC BY-NC-SA 4.0 (https://creativecommons.org/licenses/by-nc-sa/4.0/)
%
%!TEX program = xelatex
% NOTE: this template must be compiled with XeLaTeX rather than PDFLaTeX
% due to the custom fonts used. The line above should ensure this happens
% automatically, but if it doesn't, your LaTeX editor should have a simple toggle
% to switch to using XeLaTeX.
%
%%%%%%%%%%%%%%%%%%%%%%%%%%%%%%%%%%%%%%%%%

%----------------------------------------------------------------------------------------
%	PACKAGES AND OTHER DOCUMENT CONFIGURATIONS
%----------------------------------------------------------------------------------------

\documentclass[
  9pt, % Default font size, can be between 8pt and 12pt
]{FreemanCV}

\usepackage{tabularx}
\usepackage{multirow}
\usepackage{enumitem}

\renewcommand{\footnotesize}{\fontsize{6pt}{6pt}\selectfont}

\columnratio{0.55, 0.45} % Widths of the two columns, specified here as a ratio summing to 1 to correspond to percentages; adjust as needed for your content

% Headers and footers can be added with the following commands: \lhead{}, \rhead{}, \lfoot{} and \rfoot{}
% Example right footer:
%\rfoot{\textcolor{headings}{\sffamily Last update: \today. Typeset with Xe\LaTeX}}

%----------------------------------------------------------------------------------------

\begin{document}

\begin{tabularx}{1\textwidth}{
  >{\setlength\hsize{1.65\hsize}\raggedleft\arraybackslash}X
  @{\hspace{.5\columnsep}} | @{\hspace{.5\columnsep}}
  >{\setlength\hsize{0.082\hsize}\centering\arraybackslash}X
  @{\hspace{\tabcolsep}}
  >{\setlength\hsize{1.268\hsize}\raggedright\arraybackslash}X
}
  \multirow{4}{*}{\hfill\bfseries\fontsize{30pt}{36pt}\selectfont Gene Harvey}
    & \raisebox{-1pt}{\faHome} & 9880 Crestwood Ter., Eden Prairie, MN 55347 \\
    & \raisebox{-1pt}{\faPhone} & +1 (952) 999-6539 \\
    & \raisebox{-1pt}{\small\faEnvelope} & \href{mailto:gharveymn@gmail.com}{gharveymn@gmail.com} \\
    & \raisebox{-1pt}{\faGithub} & \href{https://github.com/gharveymn}{https://github.com/gharveymn} \\
\end{tabularx}

\bigskip % Vertical whitespace

\vfill % Push content to the top of the box

\begin{paracol}{2} % Begin two-column mode

%----------------------------------------------------------------------------------------
%	WORK EXPERIENCE
%----------------------------------------------------------------------------------------

\section{Work Experience}

% Each job is added with a \jobentry command. Below is an empty one to use as a template:

%\jobentry
%	{} % Duration
%	{} % Employer
%	{} % Job title
%	{} % Description

% All 5 parameters must be supplied but any can be empty if you don't need them

%------------------------------------------------

\jobentry
  {May 2019 -- Present}
  {}
  {Open Source Software Developer}
  {Created projects written in C++ and CMake, and contributed to various other projects.
   See \hyperref[sec:Open Source]{Personal Open Source Projects} for a selection of
   personal projects.}

\jobentry
  {Sep 2017 -- Dec 2017} % Duration
  {University of Manitoba} % Employer
  {Student Research Assistant} % Job title
  {Developed a multiprocess shared memory subsystem for MATLAB as well as a multi-threaded partial loader for MATLAB 7.3 HDF5 MAT-files.}

%------------------------------------------------

\jobentry
  {May 2017 -- Aug 2017} % Duration
  {NQube} % Employer
  {Consultant} % Job title
  {Worked on code optimization in automated trading models. Increased computation performance in some areas by up to 90\% with vectorization in MATLAB. Other work involved evaluation of neural network based models.}

%------------------------------------------------

\jobentry
  {May 2017 -- Sep 2017} % Duration
  {University of Manitoba} % Employer
  {Undergraduate Researcher} % Job title
  {Developed finite difference solvers for the Navier-Stokes equations in time dependent and independent forms with MATLAB. Began analysis on flow in a symmetric channel with special attention to eddy bifurcation.} % Description

%----------------------------------------------------------------------------------------
%	REFERENCES
%----------------------------------------------------------------------------------------

\section{Personal Open Source Projects}
\label{sec:Open Source}

Other projects may be found on my
\href{https://github.com/gharveymn?tab=repositories}{Github profile} which meet the same
standard of code quality, but are not fully documented.

\medskip

\begin{project}{matshare}{C}{https://github.com/gharveymn/matshare}
  \item Created a subsystem for multi-process shared memory in MATLAB.
  \item Implemented mechanisms for automatic garbage collection, thread-safe
        memory creation, and in-place overwriting.
\end{project}

 \begin{project}{gch::small\_vector}{C++}{https://github.com/gharveymn/small\_vector}
   \item Wrote a fully featured single header library implementing a vector container with a small buffer optimization.
   \item Performance on par with the implementation by Boost.
 \end{project}

\begin{project}{GNU Octave - CMake Build}{CMake}{https://sourceforge.net/projects/gnu-octave-cmake-build}
  \item Translated the build system of GNU Octave from Autotools to CMake.
  \item Resulted in a large performance increase when configuring the project
        and allowed for other features like unity builds.
\end{project}

\begin{project}{octave-ir}{C++}{https://github.com/gharveymn/octave-ir}
  \item In the process of writing a just-in-time compiler targeting the Octave/MATLAB language.
  \item Can generate code for complex expressions like nested loops and uninitialized/indeterminate
        variables.
\end{project}

\begin{project}{MEX Function Templates}{C\#}{https://marketplace.visualstudio.com/items?itemName=gharveymn.MEXFunctionTemplates}
  \item Created an extension for Visual Studio providing templates for creating MEX function projects.
  \item Over 18,000 installs, and the endorsement of an engineer at MathWorks
\end{project}

\begin{project}{NightSwitch}{TypeScript}{https://marketplace.visualstudio.com/items?itemName=gharveymn.nightswitch}
  \item Wrote a simple Visual Studio Code extension which switches between user specified
        day and night themes at sunset and sunrise.
\end{project}

\medskip % Extra vertical whitespace before the next section

%----------------------------------------------------------------------------------------

\switchcolumn % Switch to the second (right) column

\section{Technical Knowledge}

% This section is laid out using a table. A \tableentry command adds lines with the following parameters:

%\tableentry{Heading}{Content}{spaceafter}
% All 3 parameters must be supplied but any can be empty if you don't need them
% A "spaceafter" value in the third parameter will add some vertical space -- this is to be used between headings, leave it empty for no extra space

%------------------------------------------------

\begin{supertabular}{r l} % Start a table with two columns, the table will ensure everything is aligned

  %------------------------------------------------

  \tableentry{Proficient}{C++, C, CMake, MATLAB/Octave}{spaceafter}

  %------------------------------------------------

  \tableentry{Adept}{\LaTeX{}, Python, Java, C\#, Autotools}{spaceafter}

  %------------------------------------------------

  \tableentry{Beginner}{Bash, TypeScript}{spaceafter}

  %------------------------------------------------

\end{supertabular}

\section{Working Environment and Tools}

\begin{supertabular}{r l} % Start a table with two columns, the table will ensure everything is aligned

  %------------------------------------------------

  \tableentry{Primary}{Microsoft Windows, CLion,}{}
  \tableentry{}{MSYS2-MinGW, Git, CTest}{spaceafter}

  %------------------------------------------------

  \tableentry{Secondary}{Arch Linux, Zsh, Visual Studio Code,}{}
  \tableentry{}{Visual Studio}{spaceafter}

  %------------------------------------------------

\end{supertabular}

%----------------------------------------------------------------------------------------
%	EDUCATION
%----------------------------------------------------------------------------------------

\section{Education}

% Each qualification entry is added with a \qualificationentry command. Below is an empty one to use as a template:

%\qualificationentry
%	{} % Duration
%	{} % Degree
%	{} % Honors, achievements or distinctions (e.g. first class honors)
%	{} % Department
%	{} % Institution

% All 5 parameters must be supplied but any can be empty if you don't need them

%------------------------------------------------

\begin{supertabular}{r l} % Start a table with two columns, the table will ensure everything is aligned

  %------------------------------------------------

  \qualificationentry
    {2015 -- 2019} % Duration
    {Bachelor of Science} % Degree
    {} % Honors, achievements or distinctions (e.g. first class honors)
    {General Science (Physics)} % Department
    {University of Manitoba} % Institution

  %------------------------------------------------

\end{supertabular}

\section{Course Projects}

\paper
  {Efficient Path Optimization with Quadtree Partitioning}
  {Continuous empty space is converted into a grid using the quadtree data structure.
   Approximate fastest paths are then efficiently found by treating the grid as a
   weighted graph.}

\paper
  {Linear Algebra in Non-Archimedean Fields}
  {Lecture notes with introductory theorems and results regarding linear algebra
   in non-Archimedean Fields. Intended for teaching of an introductory course on
   the topic.}

\paper
  {Analysis of Image Projection and Recognition Algorithms}
  {Data collected from the eigenface algorithm is analyzed to understand the
   behavior and meaning of images at the limits of recognition set by certain
   input parameters.}

%----------------------------------------------------------------------------------------
%	SKILLS DESCRIPTION
%----------------------------------------------------------------------------------------

\section{Contributions to Other Projects}

\begin{supertabular}{r >{\raggedright\arraybackslash}p{.65\linewidth}}
  %------------------------------------------------

  \tableentry{Maintainer}{\textbf{mingw-w64-sundials}}{}
  \tableentry{}{The MSYS2-MinGW package providing the SUNDIALS library.}{spaceafter}

  %------------------------------------------------

  \tableentry{Contributer}{\textbf{GNU Octave}}{}
  \tableentry{}{Various bug fixes.}{spaceafter}

  %------------------------------------------------

\end{supertabular}
%\footnotetext{\url{https://packages.msys2.org/base/mingw-w64-sundials}}


%----------------------------------------------------------------------------------------
%	AWARDS
%----------------------------------------------------------------------------------------

\section{Awards and Scholarships}

% This section is laid out using a table. A \tableentry command adds lines with the following parameters:

%\tableentry{Heading}{Content}{spaceafter}
% All 3 parameters must be supplied but any can be empty if you don't need them
% A "spaceafter" value in the third parameter will add some vertical space -- this is to be used between headings, leave it empty for no extra space

%------------------------------------------------

\begin{supertabular}{r l} % Start a table with two columns, the table will ensure everything is aligned

  %------------------------------------------------

  \tableentry{2017}{\textbf{Guenter Krause Award}}{}
  \tableentry{2017}{\textbf{Undergraduate Research Award}}{}
  \tableentry{2016}{\textbf{Centennial Scholarship in Physics}}{}
  \tableentry{2016}{\textbf{Harold R. Coish Memorial Scholarship}}{}
  \tableentry{2016}{\textbf{UMSU Scholarship}}{spaceafter}

  %------------------------------------------------

\end{supertabular}

\section{Volunteering}

\jobentry
  {2016-2019} % Duration
  {} % Employer
  {\normalsize Physics Tutor} % Job title
  {Tutored students in first and second year physics courses.}

\jobentry
  {2016-2019} % Duration
  {} % Employer
  {\normalsize Website Administrator} % Job title
  {Created and maintained a website for the Organization of Physics Undergraduates at University of Manitoba.}

\jobentry
  {2017} % Duration
  {} % Employer
  {\normalsize Science Rendezvous Volunteer} % Job title
  {Hosted a physics demonstration booth at a science outreach event.}

%----------------------------------------------------------------------------------------

\end{paracol} % End two-column mode

%----------------------------------------------------------------------------------------

\end{document}
