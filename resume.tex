%%%%%%%%%%%%%%%%%%%%%%%%%%%%%%%%%%%%%%%%%
% Freeman Curriculum Vitae
% XeLaTeX Template
% Version 3.0 (September 3, 2021)
%
% This template originates from:
% https://www.LaTeXTemplates.com
%
% Authors:
% Vel (vel@LaTeXTemplates.com)
% Alessandro Plasmati
%
% License:
% CC BY-NC-SA 4.0 (https://creativecommons.org/licenses/by-nc-sa/4.0/)
%
%!TEX program = xelatex
% NOTE: this template must be compiled with XeLaTeX rather than PDFLaTeX
% due to the custom fonts used. The line above should ensure this happens
% automatically, but if it doesn't, your LaTeX editor should have a simple toggle
% to switch to using XeLaTeX.
%
%%%%%%%%%%%%%%%%%%%%%%%%%%%%%%%%%%%%%%%%%

%----------------------------------------------------------------------------------------
%	PACKAGES AND OTHER DOCUMENT CONFIGURATIONS
%----------------------------------------------------------------------------------------

\documentclass[
  9pt, % Default font size, can be between 8pt and 12pt
]{FreemanCV}

\usepackage{tabularx}
\usepackage{multirow}
\usepackage{enumitem}

\renewcommand{\footnotesize}{\fontsize{6pt}{6pt}\selectfont}

\columnratio{0.55, 0.45} % Widths of the two columns, specified here as a ratio summing to 1 to correspond to percentages; adjust as needed for your content

% Headers and footers can be added with the following commands: \lhead{}, \rhead{}, \lfoot{} and \rfoot{}
% Example right footer:
%\rfoot{\textcolor{footers}{
%  \begin{tabular}{r|l}
%    \raisebox{-1pt}{\today} & \raisebox{-1pt}{Typeset with \XeLaTeX}
%  \end{tabular}
%}}

%----------------------------------------------------------------------------------------

\begin{document}

\begin{tabularx}{\textwidth}{
  >{\setlength\hsize{1.65\hsize}\raggedleft\arraybackslash}X
  @{\hspace{.5\columnsep}} | @{\hspace{.5\columnsep}}
  >{\setlength\hsize{0.082\hsize}\centering\arraybackslash}X
  @{\hspace{\tabcolsep}}
  >{\setlength\hsize{1.268\hsize}\raggedright\arraybackslash}X
}
  \multirow{4}{*}{\hfill\bfseries\fontsize{30pt}{36pt}\selectfont Gene Harvey}
    & \raisebox{-1pt}{\faHome} & 1919 Franklin Ave SE, Minneapolis, MN 55414 \\
    & \raisebox{-1pt}{\faPhone} & \href{tel:19529996539}{+1 (952) 999-6539} \\
    & \raisebox{-1pt}{\small\faEnvelope} & \href{mailto:gharveymn@gmail.com}{gharveymn@gmail.com} \\
    & \raisebox{-1pt}{\faGithub} & \href{https://github.com/gharveymn}{https://github.com/gharveymn} \\
\end{tabularx}

\bigskip % Vertical whitespace

\vfill % Push content to the top of the box

\begin{paracol}{2} % Begin two-column mode

%----------------------------------------------------------------------------------------
%	WORK EXPERIENCE
%----------------------------------------------------------------------------------------

\section{Work Experience}


\begin{employer}{SmartThings}{Apr 2022 -- Present}
  \begin{jobentry}{
    \jobtitle{Software Engineer}{Sep 2022 -- Present}
    \jobtitle{Associate Software Engineer}{Apr 2022 -- Sep 2022}
  }
    Contributed to the development of protocols for the first Matter-certified
    platform on the market though efficient code output (Rust, C++), excellent feedback to
    peers, and innovative ideas for testing.\vspace{1ex}

    \begin{itemize}[
      itemsep=0.75ex,
      parsep=0ex,
      topsep=0pt,
      labelindent=0.5em,
      leftmargin=*,
      font=\normalsize
    ]
      \item Led protocols team to implement fundamental changes to architecture which
            received highly positive feedback from peers and management.
      \item Discovered root-cause and provided solutions for several complex deadlock and
            racing bugs through meticulous analysis.
      \item Developed solutions for more comprehensive testing which led to the identification of
            several critical bugs, and improved confidence in code correctness.
      \item Consistently provided consequential and actionable feedback during code reviews,
            often catching critical bugs which had gone unnoticed.
      \item Efficiently and collaboratively applied feedback from peers.
      \item Discovered and voluntarily took on additional responsibilities to resolve
            unforeseen issues.
      \item Recipient of Living Our Values employee recognition award for Q1 2023.
    \end{itemize}
  \end{jobentry}
\end{employer}

\begin{jobentry}{\jobtitle{Open Source Software Developer}{2017 -- Present}}
  Created open source projects using a variety of languages and technologies, and also contributed
  to various other projects. The following is a selection of projects which I have authored.

  \setlist{labelsep=.5em}
  \begin{description}[
    noitemsep,
    topsep=0pt,
    labelindent=1em,
    leftmargin=*,
    labelsep=0pt,
    rightmargin=1em
  ]
    \item
    \begin{project*}{matshare}{C}{https://github.com/gharveymn/matshare}
      \item Created a subsystem for multi-process shared memory in MATLAB.
      \item Implemented mechanisms for automatic garbage collection, thread-safe
            memory creation, and in-place overwriting.
    \end{project*}

    \item
    \begin{project*}{gch::small\_vector}{C++}{https://github.com/gharveymn/small\_vector}
      \item Wrote a fully featured single header library implementing a vector container with a small buffer optimization.
      \item Enforces concept requirements and supports constexpr with C++20.
      \item Performance on par with the implementation by Boost.
    \end{project*}
  \end{description}
\end{jobentry}

\begin{employer}{University of Manitoba}{}
  \begin{jobentry}{\jobtitle{Student Research Assistant}{Sep 2017 -- Dec 2017}}
    Developed a multi-process shared memory subsystem for MATLAB as well as a multi-threaded
    partial loader for MATLAB 7.3 HDF5 MAT-files.
  \end{jobentry}
\end{employer}

\begin{employer}{NQube}{}
  \begin{jobentry}{\jobtitle{Consultant}{May 2017 -- Aug 2017}}
    Worked on code optimization in automated trading models. Increased computation performance
    in some areas by up to 90\% with vectorization in MATLAB. Other work involved evaluation of
    neural network based models.
  \end{jobentry}
\end{employer}

\begin{employer}{University of Manitoba}{}
  \begin{jobentry}{\jobtitle{Undergraduate Researcher}{May 2017 -- Sep 2017}}
    Developed finite difference solvers for the Navier-Stokes equations in time dependent and
    independent forms with MATLAB. Began analysis on flow in a symmetric channel with special
    attention to eddy bifurcation.
  \end{jobentry}
\end{employer}



%----------------------------------------------------------------------------------------

\switchcolumn % Switch to the second (right) column

\section{Technical Knowledge}

% This section is laid out using a table. A \tableentry command adds lines with the following parameters:

%\tableentry{Heading}{Content}{spaceafter}
% All 3 parameters must be supplied but any can be empty if you don't need them
% A "spaceafter" value in the third parameter will add some vertical space -- this is to be used between headings, leave it empty for no extra space

%------------------------------------------------

\begin{tabular}{r l} % Start a table with two columns, the table will ensure everything is aligned

  %------------------------------------------------

  \tableentry{Proficient}{C++, Rust, C, CMake, MATLAB/Octave}{spaceafter}

  %------------------------------------------------

  \tableentry{Adept}{\LaTeX{}, Python, Java, C\#, Autotools}{spaceafter}

  %------------------------------------------------

  \tableentry{Beginner}{Bash, TypeScript}{spaceafter}

  %------------------------------------------------

\end{tabular}

\medskip

\section{Working Environment and Tools}

\begin{tabular}{r l} % Start a table with two columns, the table will ensure everything is aligned

  %------------------------------------------------

  \tableentry{Primary}{Microsoft Windows, CLion,}{}
  \tableentry{}{MSYS2-MinGW, Git, CTest}{spaceafter}

  %------------------------------------------------

  \tableentry{Secondary}{Arch Linux, Visual Studio Code,}{}
  \tableentry{}{Visual Studio}{spaceafter}

  %------------------------------------------------

\end{tabular}

\medskip

%----------------------------------------------------------------------------------------
%	EDUCATION
%----------------------------------------------------------------------------------------

\section{Education}

% Each qualification entry is added with a \qualificationentry command. Below is an empty one to use as a template:

%\qualificationentry
%	{} % Duration
%	{} % Degree
%	{} % Honors, achievements or distinctions (e.g. first class honors)
%	{} % Department
%	{} % Institution

% All 5 parameters must be supplied but any can be empty if you don't need them

%------------------------------------------------

\begin{tabular}{r l} % Start a table with two columns, the table will ensure everything is aligned

  %------------------------------------------------

  \qualificationentry
    {2015 -- 2019} % Duration
    {Bachelor of Science} % Degree
    {} % Honors, achievements or distinctions (e.g. first class honors)
    {General Science (Physics)} % Department
    {University of Manitoba} % Institution

  %------------------------------------------------

\end{tabular}

\medskip

\section{Course Projects}

\paper
  {Efficient Path Optimization with Quadtree Partitioning}
  {Continuous empty space is converted into a grid using the quadtree data structure.
   Approximate fastest paths are then efficiently found by treating the grid as a
   weighted graph.}

\paper
  {Linear Algebra in Non-Archimedean Fields}
  {Lecture notes with introductory theorems and results regarding linear algebra
   in non-Archimedean Fields. Intended for teaching of an introductory course on
   the topic.}

\paper
  {Analysis of Image Projection and Recognition Algorithms}
  {Data collected from the eigenface algorithm is analyzed to understand the
   behavior and meaning of images at the limits of recognition set by certain
   input parameters.}

%----------------------------------------------------------------------------------------
%	AWARDS
%----------------------------------------------------------------------------------------

\section{Awards and Scholarships}

% This section is laid out using a table. A \tableentry command adds lines with the following parameters:

%\tableentry{Heading}{Content}{spaceafter}
% All 3 parameters must be supplied but any can be empty if you don't need them
% A "spaceafter" value in the third parameter will add some vertical space -- this is to be used between headings, leave it empty for no extra space

%------------------------------------------------

\begin{tabular}{r l} % Start a table with two columns, the table will ensure everything is aligned

  %------------------------------------------------

  \tableentry{2017}{\textbf{Guenter Krause Award}}{}
  \tableentry{2017}{\textbf{Undergraduate Research Award}}{}
  \tableentry{2016}{\textbf{Centennial Scholarship in Physics}}{}
  \tableentry{2016}{\textbf{Harold R. Coish Memorial Scholarship}}{}
  \tableentry{2016}{\textbf{UMSU Scholarship}}{spaceafter}

  %------------------------------------------------

\end{tabular}

\section{Contributions to Other Projects}

\begin{tabular}{r >{\raggedright\arraybackslash}p{.65\linewidth}}

  %------------------------------------------------
  %\textsc{Maintainer} & \textbf{mingw-w64-sundials} \\[0.5ex]
  %& The MSYS2-MinGW package providing the SUNDIALS library. \\[1ex]
  %\textsc{Contributor} & \textbf{GNU Octave} \\[0.5ex]
  %& Various bug fixes. \\[1ex]

  \tableentry{Maintainer}{\textbf{mingw-w64-sundials}}{}
  \tableentry{}{The MSYS2-MinGW package providing the SUNDIALS library.}{spaceafter}
  & \\[-1.5ex] % idk latex is being weird and annoying.

  %------------------------------------------------

  \tableentry{Contributor}{\textbf{GNU Octave}}{}
  \tableentry{}{Various bug fixes.}{}

  %------------------------------------------------

\end{tabular}
%\footnotetext{\url{https://packages.msys2.org/base/mingw-w64-sundials}}
%
%----------------------------------------------------------------------------------------
%
\end{paracol} % End two-column mode
%
%----------------------------------------------------------------------------------------
%
\end{document}
